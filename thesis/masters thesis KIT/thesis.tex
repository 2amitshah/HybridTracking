\documentclass{thesisclass}
% Based on thesisclass.cls of Timo Rohrberg, 2009
% ----------------------------------------------------------------
% Thesis - Main document
% ----------------------------------------------------------------


%% -------------------------------
%% |  Information for PDF file   |
%% -------------------------------
\hypersetup{
 pdfauthor={Felix Achilles},
 pdftitle={Masters Thesis Achilles - Hybrid Tracking},
 pdfsubject={thesis},
 pdfkeywords={thesis, tracking, Kalman, unscented, fusion, NDI, navigation}
}


%% ---------------------------------
%% | Information about the thesis  |
%% ---------------------------------

\newcommand{\myname}{Felix Achilles}
\newcommand{\mytitle}{Hybrid Tracking\\ 
					  Multimudal Fusion of Tracking Streams using a Federated Unscented Kalman Filter}
\newcommand{\myinstitute}{TU Munich\\
					      KIT Karlsruhe}

\newcommand{\reviewerone}{Nassir Navab}
\newcommand{\reviewertwo}{Olaf Dössel}
\newcommand{\advisor}{Bernhard Fürst}
\newcommand{\advisortwo}{Amit Shah}

\newcommand{\timestart}{01. May 2013}
\newcommand{\timeend}{01. Nov 2013}
\newcommand{\submissiontime}{01. 11. 2013}


%% ---------------------------------
%% | ToDo Marker - only for draft! |
%% ---------------------------------
% Remove this section for final version!
\setlength{\marginparwidth}{20mm}

\newcommand{\margtodo}
{\marginpar{\textbf{\textcolor{red}{ToDo}}}{}}

\newcommand{\todo}[1]
{{\textbf{\textcolor{red}{(\margtodo{}#1)}}}{}}


%% --------------------------------
%% | Old Marker - only for draft! |
%% --------------------------------
% Remove this section for final version!
\newenvironment{deprecated}
{\begin{color}{gray}}
{\end{color}}


%% --------------------------------
%% | Settings for word separation |
%% --------------------------------
% Help for separation:
% In german package the following hints are additionally available:
% "- = Additional separation
% "| = Suppress ligation and possible separation (e.g. Schaf"|fell)
% "~ = Hyphenation without separation (e.g. bergauf und "~ab)
% "= = Hyphenation with separation before and after
% "" = Separation without a hyphenation (e.g. und/""oder)

% Describe separation hints here:
\hyphenation{
% Pro-to-koll-in-stan-zen
% Ma-na-ge-ment  Netz-werk-ele-men-ten
% Netz-werk Netz-werk-re-ser-vie-rung
% Netz-werk-adap-ter Fein-ju-stier-ung
% Da-ten-strom-spe-zi-fi-ka-tion Pa-ket-rumpf
% Kon-troll-in-stanz
}


%% ------------------------
%% |    Including files   |
%% ------------------------
% Only files listed here will be included!
% Userful command for partially translating the document (for bug-fixing e.g.)
\includeonly{%
titlepage,
declaration,
introduction,
content,
evaluation,
conclusion,
appendix
}


%%%%%%%%%%%%%%%%%%%%%%%%%%%%%%%%%
%% Here, main documents begins %%
%%%%%%%%%%%%%%%%%%%%%%%%%%%%%%%%%
\begin{document}

% Remove the following line for German text
\selectlanguage{english}

\frontmatter
\pagenumbering{roman}
\include{titlepage}
\include{declaration}
\blankpage


%% -------------------
%% |   Directories   |
%% -------------------
\tableofcontents
\blankpage


%% -----------------
%% |   Main part   |
%% -----------------
\mainmatter
\pagenumbering{arabic}
%% introduction.tex
%%

%% ==============================
\chapter{Introduction}
\label{ch:Introduction}
%% ==============================

\cite{becker2008a}
\dots


%% content.tex
%%

%% ==============
\chapter{Content Chapter 1}
\label{ch:Content1}
%% ==============

The content chapters of your thesis should of course be renamed. How many chapters you need to write depends on your thesis and cannot be said in general. 

Check our the examples theses in the Wiki. 

Of course, you can split this .tex file into several files if you prefer. 


%% ===========================
\section{Section 1}
\label{ch:Content1:sec:Section1}
%% ===========================

\dots


%% ===========================
\section{Section 2}
\label{ch:Content1:sec:Section2}
%% ===========================

\dots


%% content.tex
%%

%% ==============
\chapter{Content Chapter 2}
\label{ch:Content1}
%% ==============

\dots


%% ===========================
\section{Section 1}
\label{ch:Content2:sec:Section1}
%% ===========================

\dots


%% ===========================
\section{Section 2}
\label{ch:Content2:sec:Section2}
%% ===========================

\dots

Add additional content chapters if required. 
\include{evaluation}
\include{conclusion}


%% --------------------
%% |   Bibliography   |
%% --------------------
\cleardoublepage
\phantomsection
\addcontentsline{toc}{chapter}{\bibname}

\iflanguage{english}
{\bibliographystyle{IEEEtranSA}}	% english style
{\bibliographystyle{babalpha-fl}}	% german style
												  
% Use IEEEtran for numeric references
%\bibliographystyle{IEEEtranSA})

\bibliography{thesis}


%% ----------------
%% |   Appendix   |
%% ----------------
\cleardoublepage

\input{appendix}


\end{document}
